\documentclass{article}
\usepackage[margin=1in]{geometry}%设置边距,符合Word设定
\usepackage{ctex}
\usepackage{setspace}
\usepackage{lipsum}
\usepackage[all]{xy}

\title{\heiti\zihao{2} Homework 17}
\author{\songti 231275040 林方恒}

\begin{document}
	\maketitle
	\thispagestyle{empty}

\section{Problem 1}\par
令$\langle D_{12}, \mid \rangle$表示12的所有正因子组成的偏序集。\\
(1) 证明 $\langle D_{12}, \mid \rangle$ 是一个偏序格,并由此定义运算∗和◦,证明$\langle D_{12}, \mid \rangle$是对应的代数格\\

a.由于$x \vee y = lcm(x, y)$, $x \wedge y = gcd(x, y)$, 
由于$gcd(x, y)$, $lcm(x, y)$始终存在,
令$x \in D_{12},\ y \in D_{12},\ z \in D_{12}$, 
有$z \mid x,\ z \mid y $, 故$z \mid gcd(x, y)$, 即$gcd(x, y)$为最大下界.
同理, $lcm(x, y)$为最小上界, 所以$<D_{12}, |>$是一个偏序格.\par
b.定义运算$\ast = lcm$和$\circ = gcd$, 显然满足交换律和结合律, 
$ x \ast (x \circ y) = lcm(x, gcd(x, y)) = x,\ x \circ (x \ast y) = gcd(x, lcm(x, y)) = y$
满足吸收律, 故其为代数格.

~\\
(2) 按照 (1) 的定义,说明 $\langle D_{12},\ \ast,\ \circ \rangle$ 是否是一个有补格\par
如图, 显然2, 6没有补元
$$
\def\arl{\ar@{-}}
\xymatrix{
  &  &  12\arl[dl]\arl[dr]  &  \\
  &  6\arl[dl]\arl[dr] &  &4\arl[dl]\\
  3\arl[dr] & &2\arl[dl] & \\
  &1&&\\
}
$$

~\\
(3) 按照 (1) 的定义,说明 $\langle D_{12},\ \ast,\ \circ \rangle$ 是否是一个分配格

如图, 此格的任意子格均不同构于$M_3$或$M_5$, 故其为分配格.
\newpage
\section{Problem 2}\par
下列各集合对于整除关系都构成偏序集,判断哪些偏序集是格.\\
(1) $L = \{1,\ 2,\ 3,\ 4,\ 5\};$\\
(2) $L = \{1,\ 2,\ 3,\ 6,\ 12\};$\\
(3) $L = \{1,\ 2,\ 3,\ 4,\ 6,\ 9,\ 12,\ 18,\ 36\};$\\
(4) $L = \{1,\ 2,\ 22 ,\ · · · ,\ 2n ,\ · · · \};$\\

\end{document}