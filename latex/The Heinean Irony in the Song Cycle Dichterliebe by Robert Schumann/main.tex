%%%%%%%%%%%%%%%%%%%%%%%%%%%%%%%%%%%%%%%%%%%%%%%%%%%%%%%%%%%
% --------------------------------------------------------
% Rho
% LaTeX Template
% Version 1.0 (28/04/2024)
%
% Author: 
% Guillermo Jimenez (memo.notess1@gmail.com)
% 
% License:
% Creative Commons CC BY 4.0
% --------------------------------------------------------
%%%%%%%%%%%%%%%%%%%%%%%%%%%%%%%%%%%%%%%%%%%%%%%%%%%%%%%%%%%
% --------------------------------------------------------
%					  FOR SPANISH BABEL
% --------------------------------------------------------
% \usepackage[spanish,es-nodecimaldot,es-noindentfirst]{babel}
% --------------------------------------------------------
%%%%%%%%%%%%%%%%%%%%%%%%%%%%%%%%%%%%%%%%%%%%%%%%%%%%%%%%%%%

\documentclass[10pt,a4paper,onecolumn]{rho}
\usepackage[english]{babel}
\usepackage{rhoenvs}




%----------------------------------------------------------
% TITLE
%----------------------------------------------------------

\title{The Heinean Irony in the Song Cycle \textit{Dichterliebe} by Robert Schumann}

%----------------------------------------------------------
% AUTHORS AND AFFILIATIONS
%----------------------------------------------------------

\author{Henry Lin(Fangheng Lin), \textit{Computer Science and Financial Engineering, Nanjing University}}


%----------------------------------------------------------
\affil{\quad NJU mail: 231275040@smail.nju.edu.cn}

%----------------------------------------------------------
% DATES
%----------------------------------------------------------


%----------------------------------------------------------
% FOOTER INFORMATION
%----------------------------------------------------------
\smalltitle{The Heinean Irony in the Song Cycle \textit{Dichterliebe} by Robert Schumann}
\etal{Henry}
\footinfo{ }
\institution{Nanjing University}
\date{today}

%----------------------------------------------------------
% ARTICLE INFORMATION
%----------------------------------------------------------

\email{\\
\quad QQ; 3351397126@qq.com\\
\quad Gmail; linhenry980@gmail.com\\
\quad NJU mail: 231275040@smail.nju.edu.cn}
\license{Henry Lin \ccLogo \ student id: 231275040}

%----------------------------------------------------------
% ABSTRACT
%----------------------------------------------------------

\begin{abstract}
Heinrich Heine, as a renowned German poet and writer, has always been a popular subject of literary research. However, another aspect of Heine's identity, often overlooked (especially domestically), is that of a lyricist. Schumann, as an important composer of German Romanticism, had a deep connection with literature and even engaged in literary creation himself. This paper aims to start from the dual identities of Heine and Schumann, using Schumann's \textit{"Dichterliebe,"} based on Heine's \textit{"Buch der Lieder,"} as the research object. It selects "Heinesche Ironie" (Heinean irony) as the connecting point for textual and musical analysis, attempting to prove that the "ironic" elements in Heine's original poems are reproduced in different forms in Schumann's music, achieving a perfect fusion of text and music in this song cycle.
\end{abstract}


%----------------------------------------------------------

\keywords{\textit{Dichterliebe}, Art Song, Literature-Music Relations}

%----------------------------------------------------------
\geometry{left=1cm, right=1cm, top=1.5cm, bottom=1.5cm}
\begin{document}
	

    \maketitle
    \thispagestyle{firststyle}
    \tableofcontents

%----------------------------------------------------------
    \newgeometry{left=6cm, right=1cm, top=1.5cm, bottom=1.5cm}
    \onecolumn
\section{Introduction}
    \rhostart{T}he song cycle \textit{Dichterliebe} by the Romantic composer Robert Schumann is one of the most famous song cycles in music history. Due to its beautiful melody (among other reasons), this masterpiece remains a "regular guest" at concerts. The popularity of the cycle "is likely surpassed only by Schubert's \textit{Winterreise} in Germany, but by no other German song cycle abroad" . Moreover, it is also a subject of study because of its high literary quality and musical aesthetics. The text source comes from Heinrich Heine's \textit{Book of Songs} (section \textit{Lyrisches Intermezzo}), a collection of poems and ballads published in 1824. It deals with disappointed love, longing hope, and simple folk beliefs. As Albrecht Dümling notes, the \textit{Book of Songs} then literally became a book of songs. Some songs, such as\textit{ "The Lorelei,"} became so well known that people even forgot who originally wrote the lyrics.

\section{Abstract}
    \rhostart{T}he paper is divided into six chapters. The first chapter serves as an introduction. The second chapter explains two basic concepts involved in this paper: "art song" (the genre of \textit{Dichterliebe}) and the definition of "Heinean irony" in the literary domain. In the third chapter, the author introduces Schumann's dual talent in music and literature, as well as the role of literature in his musical creation, followed by a discussion on the common aesthetic points between Schumann and Heine. The fourth chapter delves into the work \textit{Dichterliebe}, focusing on how Schumann chose and arranged the texts for his song cycle. The fifth chapter is the main part of this paper, providing a textual and musical analysis based on six selected songs. The author believes that "Heinean irony" in this song cycle is reflected in two dimensions: firstly, the basic form of irony, which lies in the difference between what is said and what is meant (Differenzen zwischen Gesagtem und Gemeintem), and secondly, in the contrast between dreams and reality, where the dream motif in Heine's original poems is always connected to the reality upon awakening, with the huge contrast constituting another dimension of the irony in the poetry collection. When analyzing the songs, the author first examines the text to identify the ironic elements in Heine's original poems, then points out the modifications Schumann made during the composition process and their underlying intentions. Following this, the author analyzes the musical reproduction of irony through the choice of key, rhythm settings, coordination between the vocal and accompaniment parts, and the coda's musical figures. The paper concludes that this vocal cycle is not "overly gentle" as previous studies have suggested, but rather fully expresses the "irony" present in Heine's original poems.

\section{Conceptual Explanation}
\rhostart{I}n this section, some fundamental concepts will be explained: the art song and its significance in research concerning the border area between literature and music, irony and Heine's irony in the literary sense, as well as their forms of expression.

\subsection{Kunst Lied(Art Song)}
 \rhostart{L}ied is a term used in both musicology and literary studies. In the Lied, text and melody are combined. There are various types of Lieder, such as folk songs, hymns, and art songs. The songs discussed below belong to the category of art songs. This genre holds a special place in German Romanticism and, in a certain sense, embodies the ideal of Romantic poetics: different art forms unite to form "a new whole of art" Thanks to the contributions of famous composers such as Franz Schubert and Robert Schumann, German poetry achieved worldwide recognition and remains influential today.

In a narrower sense, the art song is a European Lied. This genre emerged in 1810 with Franz Schubert's compositions, underwent a process of continuous musical transformation over the century, and experienced the end of its heyday in the works of Richard Strauss and Arnold Schoenberg. Throughout this process, the focus was on German-language songs. In a broader sense, the art song is a solo song from the Western world with (chordal) instrumental accompaniment since the introduction of monody around 1600. 

Hans-Joachim Hinrichsen summarized the characteristics of the art song in his essay \textit{Das Kunstlied als musikalische Lyrik} as follows:
\begin{quote}
    \textcolor{darkgray}{"The careful selection of text templates, the idiosyncratic and melodically complex structures that break away from the traditional ideal of simplicity, an autonomously structured piano accompaniment (with the piano used as the irreplaceable instrument), a sophisticated compositional texture, a structure subtly shifting between strophic form and through-composition, and a harmony extending far beyond mere basic levels."}
\end{quote}

Two points are noteworthy here: the literary quality of the text template is important. For example, Schumann set poems by Heine, Uhland, Eichendorff, and Rückert to music. He chose these poets because their poems embodied "the new poetic spirit" and "were reflected in the music." Music, particularly the piano part, does not merely fulfill an accompanying function but plays an equally important role alongside the text. In a certain sense, the musical setting is an interpretation of the text.

\subsection{Heinean Irony}
Heine's work is characterized by the literary device of irony. He once remarked that his poems, even when "the lyrical stands out, are entirely permeated by a more intellectual element, irony." This gives rise to a specific concept: Heinean irony, which can be regarded on the one hand as a characteristic or mindset, and on the other hand as a purely literary device. In the present study, this concept is only considered in the literary sense.

Heinean irony is aimed specifically at "the dissolution of unrealistic dreams". Heine's poems are full of contrasts and ruptures, and one always encounters "ironic points, the use of banalities, the parody of topoi, and self-irony" in his works. "The moment of contrast" is of great importance to Heine, especially in the context of juxtaposing dream and reality. Ursula Lehmann, in her work \textit{Popularisierung und Ironie im Werk Heinrich Heines}, has highlighted two essential features of Heine's irony: the first feature is a disillusioning function consistent with romantic irony, and the second is the torn nature that repeatedly surfaces in his poems.

Regarding Heine's \textit{Buch der Lieder}, Erich Mayser argues that in this work, "an underlying-skeptical-ironic tone seems to resonate more or less subtly". The ironic placement in poems, according to Gerhard Höhn, serves the purpose of "creating distance and ensuring self-protection". Laughter and comedy are part of the "survival strategy of a wounded and torn individual". They are rather a mask and shell of the poet. A detailed explanation of the term with specific examples will be provided below along with text analysis.
%----------------------------------------------------------

\printbibliography[title={REFERENCES}]

%----------------------------------------------------------

\end{document}